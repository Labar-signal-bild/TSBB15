\subsection{Object Detection}
To be able to track a moving object in an image the first step is to find the object that is of interest. 

To detect all the moving objects in an image start by compute all interest points in two on each other following images. The interest points will be found by using the Harris algorithm. These points will be compared and all the corresponding image points that has moved will be studied. These points are the ones that might be an object of interest. By making the assumption that an object have many interest points, we will look for a cluster of interest points that move in the same way. Clusters of moving interest points that move randomly can be assumed to be something uninteresting. This can be for example a moving crown of a tree, hence something that is not something to track.

Another way to find objects is to do background modeling. When using background modeling pixels that differ from the "normal state" of the background will be marked with a $1$ and pixels that fit to the modeled background will be marked with $0$. This means that a binary image of differences from an all background image will be made. By processing this binary image with morphological functions objects in the image can be found. When the objects are found the interest points in just those parts of the image will be found using the Harris algorithm. 

Using background modeling to detect objects might be added in a later state. The implementation of modeling the background is not the first priority. 

The object detection module will send the pixels of the interest points that has been found to the tracking module. 
